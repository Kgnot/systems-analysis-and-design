\documentclass{article}
\usepackage{amsmath}
\usepackage[english]{babel}
\usepackage[utf8]{inputenc}
\usepackage[a4paper, margin=2.5cm]{geometry} 
\usepackage{geometry}
\usepackage{titlesec}
\usepackage{enumitem} 
\usepackage{xcolor}
\usepackage{hyperref}
%\usepackage{pdfpages}
\usepackage{graphicx}


\title{Workshop 1}

\author{Henry Ricaurte Mora 20221020084 \\ Germán Darío Aya Fuentes 20232020091 \\ Javier Alejandro Penagos Hernández 20221020028}

\date{}

\begin{document}

\maketitle


\section*{Context}

Game-based learning offers a fun and dynamic way of teaching, allowing students to interact with educational content in playful environments. The main objective is to predict student performance; specifically, the goal is to determine whether a user will answer a question correctly based on their behavior. This prediction is closely related to the level they reach in the game, which depends on how many questions they answer correctly.

Although the use of educational games has grown, there are still few open datasets that enable the application of data science to optimize these platforms. In particular, knowledge tracing—which allows for personalized learning—has been little explored in educational games.

The Kaggle competition is organized by Field Day Lab, a public lab from the Wisconsin Center for Education Research, which develops free and accessible games to study how people learn. In collaboration with organizations like The Learning Agency Lab, they aim to improve educational games and provide useful analytical tools for educators. If successful, these initiatives could expand and strengthen the use of game-based learning in formal education.

\section*{Components: }

\begin{itemize}[leftmargin=*]
    \item \textbf{\textbf{Mouse}}: It is the first input. It is necessary for the functioning of the analyzed system.
    \item \textbf{\textbf{Hover}}, \textbf{\textbf{Click}}: Functions that are part of the first input.
    \item \textbf{\textbf{X}}, \textbf{\textbf{Y}}: Cartesian location of each of the inputs within the system.
    \item \textbf{\textbf{Event\_handler}}: Group responsible for handling each of the individual events of the system.
    \item \textbf{\textbf{Name}}: Name of each event in the system.
    \item \textbf{\textbf{Text}}: Text associated with each event in the system.
    \item \textbf{\textbf{Session\_id}}: Unique identifier of the session.
    \item \textbf{\textbf{User\_id}}: Unique identifier of the user within the system.
    \item \textbf{\textbf{Level}}: User parameter that characterizes their skill level and/or progress.
    \item \textbf{\textbf{Configuration}}: Component responsible for managing the game's configurations.
    \item \textbf{\textbf{Hq}}: Identifier related to the visual configuration of the system.
    \item \textbf{\textbf{Full\_screen}}: Identifier related to the visual configuration of the system.
    \item \textbf{\textbf{Option\_music}}: Component that balances the game’s music.
    \item \textbf{\textbf{Id\_notebook}}: Unique identifier of the notebook.
    \item \textbf{\textbf{Pages}}: Space for making notes, paraphrasing ideas, or building a solution.
    \item \textbf{\textbf{Event\_registry}}: Registry of all events in the system.
    \item \textbf{Screen}: Component responsible for managing the coordinates of the system.
    \item \textbf{Room }: Component responsible for managing the coordinates of the system and helpi to creation of questions.
    \item \textbf{Question}: Management the level of the system.
    \item \textbf{\textbf{Data\_analytic}}: Component responsible for generating and managing the system’s solution to reach the objective. Its main purpose is to \textit{predict, based on player behavior, whether a user will answer a question correctly}.
    \item \textbf{\textbf{Performance}}: Main output of the system (not feedback-based). It reflects \textit{how many questions the user answered correctly and which \textbf{level\_group} they reached}.

\end{itemize}

\section*{Relationships}

\par
As a primary relationship, we have the one that generates the input: the \textbf{\textbf{mouse}}, 
which contains two main events: \textbf{\textbf{hover}} and \textbf{\textbf{click}}, where each 
of these is mapped using a Cartesian coordinate (\textbf{\textbf{x}}, \textbf{\textbf{y}}). This 
first "major" system constantly communicates by sending all information 
to the \textbf{\textbf{event\_handler}}, which contains a \textbf{\textbf{name}} and 
a \textbf{\textbf{text}} for each of these events.
\\
\\
Continuing with the design of the user interface system, after entering the
 \textbf{\textbf{user\_id}}, it obtains a \textbf{\textbf{level}}, which is responsible for 
 setting in relation to the \textbf{\textbf{user\_id}}. This same \textbf{\textbf{user\_id}} 
 component implements a \textbf{\textbf{configuration}} and creates a \textbf{\textbf{session\_id}},
  which is then responsible for sharing that information with the \textbf{\textbf{event\_handler}}.
  \\
  \\
Approaching the \textbf{\textbf{configuration}} section, we have 3 relationships that obtain state
 information and share it with the corresponding components: \textbf{\textbf{hq}},
  \textbf{\textbf{full\_screen}}, and \textbf{\textbf{option\_music}}. 
  \\
 \\
 \\
In \textbf{\textbf{user\_id}} we also handle another relationship: this one gets to 
\textbf{\textbf{id\_notebook}}’s component, which contains \textbf{\textbf{pages}} 
that are connected with an event we will manage as \textbf{\textbf{event\_handler}}.
\\
\\
\textbf{\textbf{Event\_handler}} is an important component since it adds everything it receives 
to another component called \textbf{\textbf{event\_registry}}, which is in charge of handling all 
event records. These are then \textbf{\textbf{sent\_to}} the \textbf{\textbf{data\_analystic}}’s 
component, which generates a \textbf{\textbf{performance}} of the analyzed system.

\section*{Model}
\begin{center}
    \includegraphics[width=0.75\textwidth]{src/IO.pdf}
  \end{center}

  \section*{Complexity and Sensitivity}
  Sensitivity analysis allows us to understand how changes in one component of the system can affect the entire system. To do this, we analyze how each component influences the system and to what extent, with a particular focus on those components where sensitivity is highest—i.e., where variations in the output can most significantly impact the overall system. All of this is aimed at predicting player performance based on how these components behave.\\
  
  \textbf{First, we begin with the inputs. In the system, two inputs were defined.}
  
  \subsection*{Mouse Events}
  Events refer to the actions performed by the player using the mouse. These actions have a significant impact on the system, as depending on the decisions the player makes, how they make them, and the time taken to execute them, the system will respond in different ways. While there is not complete randomness—since in games actions are predefined—the variation lies in how the player carries them out.
  \\
  
  \textbf{Now that we have analyzed the inputs, we will analyze the system components that influence the system.}
  
  \subsection*{Level}
  The level reached by the player is a key element, as it reflects the progress and skills acquired during the game. However, it may present some limitations depending on the context, since this factor also largely depends on the player's ability.
  
  \subsection*{Session Id}
  The duration of each session per player allows for the analysis of whether players quickly become engaged with the game or, on the contrary, abandon it quickly. This factor is important for predicting player performance.
  
  \subsection*{Configuration}
  Configuration is a very important aspect, as it can significantly influence the player's performance. Depending on the settings the player has adjusted, their in-game performance may vary in different ways. For the system presented, these settings include music volume, whether the game is in fullscreen mode, and the game’s quality.
  
  \subsection*{Data Analytics}
  This system component will analyze the outputs of the previously described components to identify patterns that help determine player performance. The limitations of this component are directly related to the inputs—therefore, if these are biased, the prediction will be as well.
  

  \section*{Chaos and Randomness}

  \subsection*{A Complex and Dynamic System}

Although the system appears deterministic and computational, in practice 
it behaves as a complex adaptive system. The diversity of individual 
trajectories, the influence of context, and the continuous feedback 
between the student and the environment make it a system whose evolution 
cannot be reduced to simple or linear rules.

These systems are characterized by the interaction of multiple variables, 
the emergence of behaviors, and sensitivity to small perturbations in the 
initial state. In the case of learning through games, the decisions, 
emotions, and particular conditions of each player interact with the 
system rules in unpredictable ways, generating highly variable dynamics 
that are difficult to model.

\subsection*{Chaotic Dynamics in Learning}

A deeper analysis reveals characteristics of a chaotic system, understood 
as one that, although deterministic, exhibits unpredictable behaviors due 
to its extreme sensitivity to initial conditions. This is accompanied by 
high nonlinearity, the emergence of patterns, and the presence of strange 
attractors—all key concepts from chaos theory.

\subsection*{Sensitivity to Initial Conditions}

Each student starts a game session with a set of personal and contextual 
variables: prior knowledge, motivation, emotional state, past experiences. 
Small variations in these factors can lead to completely different 
interaction trajectories with the system, making behavior very difficult 
to predict with certainty.

\subsection*{Nonlinearity}

The factors that influence performance do not have a direct or 
proportional relationship with results. For example, more time spent on a 
task may mean reflection in one case and confusion in another. This 
nonlinearity makes the system highly unpredictable and challenging to 
model.

\subsection*{Emergence and Attractors}

As students interact with the system, emergent behavior patterns appear.
Some learn through trial and error, others become frustrated and quit, and 
some internalize the rules and optimize their performance. These patterns 
can be understood as strange attractors—conceptual figures from chaos 
theory representing sets of states towards which a system tends without 
ever exactly repeating itself.

\subsection*{Chaotic Dynamics by Component}

\begin{enumerate}
    \item \textbf{Mouse Events (clicks, hover, x/y coordinates)}
    \begin{itemize}
        \item \textit{Chaotic nature:} Reflect micro-decisions by the 
        student, influenced by attention, environmental interpretation, 
        internal strategies, and emotional state.
        \item \textit{Randomness:} The same stimulus can provoke different 
        responses depending on the individual's context.
        \item \textit{Impact:} Introduces noise into the system, 
        complicating the direct association between event and performance.
    \end{itemize}
    
    \item \textbf{Response Time}
    \begin{itemize}
        \item \textit{Chaotic nature:} Affected by reading speed, fatigue,
        motivation, distractions, etc.
        \item \textit{Randomness:} A fast response may indicate confidence 
        or guessing; a slow one may indicate reflection or confusion.
        \item \textit{Impact:} Needs contextualization for correct 
        interpretation; its relation to performance is non-linear.
    \end{itemize}
    
    \item \textbf{Level (player level)}
    \begin{itemize}
        \item \textit{Chaotic nature:} Depends on the accumulation of 
        prior decisions and the player's adaptation to the environment.
        \item \textit{Randomness:} Two players at the same level may have 
        followed completely different paths.
        \item \textit{Impact:} The level indicates relative progress, but 
        not necessarily deep understanding.
    \end{itemize}
    
    \item \textbf{Session\_id (session duration and frequency)}
    \begin{itemize}
        \item \textit{Chaotic nature:} Influenced by personal and external
        factors, such as time availability or level of interest.
        \item \textit{Randomness:} Sessions of the same duration may be 
        qualitatively very different.
        \item \textit{Impact:} Affects the amount of data per player and 
        how their level of engagement is interpreted.
    \end{itemize}
    
    \item \textbf{Configuration (visual and sound options)}
    \begin{itemize}
        \item \textit{Chaotic nature:} Interface variations change the 
        sensory experience of the game.
        \item \textit{Randomness:} Nonlinear impact on perception, 
        concentration, or stress.
        \item \textit{Impact:} Introduces heterogeneity into the 
        environment that modifies player behavior.
    \end{itemize}
    
    \item \textbf{Event\_handler and Event\_registry}
    \begin{itemize}
        \item \textit{Chaotic nature:} Represent unique and unrepeatable 
        interaction sequences.
        \item \textit{Randomness:} Each user follows a distinct path, with 
        no exact replication.
        \item \textit{Impact:} Key input for modeling, but require 
        dimensionality reduction or grouping due to their complexity.
    \end{itemize}
    
    \item \textbf{Data\_analytic and Performance}
    \begin{itemize}
        \item \textit{Chaotic nature:} Dependent on the quality and 
        representativeness of the inputs.
        \item \textit{Randomness:} The model may detect misleading 
        patterns or apparent relationships that do not reflect real 
        connections if the data contains bias or noise.
        \item \textit{Impact:} Limits the model's accuracy in trying to 
        generalize in an inherently unpredictable system.
    \end{itemize}
\end{enumerate}

  \section*{Conclusion}
  The system, based on the analysis performed, demonstrates a certain robustness in its relationships. This property ensures that its results are more precise and less susceptible to errors.\\
  Although some uncertainty remains, it revolves around understanding the type of games it offers does the catalog exhibit a high and significant level of skill? These questions are necessary 
  to provide a clearer understanding of the variability in configurations \\
  We observe that the impact of each element is balanced in relation to both the system and its objective. 
  While one component stands out such as \underline{\textbf{event\_handler}} and 
  \underline{\textbf{event\_registry}} we can still note an overall balance in the importance weights.\\
  Another important point is its internal configuration. Here, we can identify small subregions that improve diagram readability regions with significantly strong influence that,
   in addition to providing balance (as previously mentioned), offer control over the problem's surface and its purpose.


\end{document}
